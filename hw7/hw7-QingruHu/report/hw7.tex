% Modified based on Xiaoming Sun's template
\documentclass{article}
\usepackage{ctex}
\usepackage{amsmath,amsfonts,amsthm,amssymb}
\usepackage{setspace}
\usepackage{fancyhdr}
\usepackage{lastpage}
\usepackage{extramarks}
\usepackage{chngpage}
\usepackage{soul,color}
\usepackage{graphicx,float,wrapfig}
\usepackage{enumitem}
\usepackage{array} 
\newcommand{\Class}{Pattern Recognition and Machine Learning}

% Homework Specific Information. Change it to your own
\newcommand{\Title}{Homework 2}

% In case you need to adjust margins:
\topmargin=-0.45in      %
\evensidemargin=0in     %
\oddsidemargin=0in      %
\textwidth=6.5in        %
\textheight=9.0in       %
\headsep=0.25in         %

% Setup the header and footer
\pagestyle{fancy}                                                       %
\chead{\Title}  %
\rhead{\firstxmark}                                                     %
\lfoot{\lastxmark}                                                      %
\cfoot{}                                                                %
\rfoot{Page\ \thepage\ of\ \protect\pageref{LastPage}}                          %
\renewcommand\headrulewidth{0.4pt}                                      %
\renewcommand\footrulewidth{0.4pt}                                      %

%%%%%%%%%%%%%%%%%%%%%%%%%%%%%%%%%%%%%%%%%%%%%%%%%%%%%%%%%%%%%
% Some tools
\newcommand{\enterProblemHeader}[1]{\nobreak\extramarks{#1}{#1 continued on next page\ldots}\nobreak%
                                    \nobreak\extramarks{#1 (continued)}{#1 continued on next page\ldots}\nobreak}%
\newcommand{\exitProblemHeader}[1]{\nobreak\extramarks{#1 (continued)}{#1 continued on next page\ldots}\nobreak%
                                   \nobreak\extramarks{#1}{}\nobreak}%

\newcommand{\homeworkProblemName}{}%
\newcounter{homeworkProblemCounter}%
\newenvironment{homeworkProblem}[1][Problem \arabic{homeworkProblemCounter}]%
  {\stepcounter{homeworkProblemCounter}%
   \renewcommand{\homeworkProblemName}{#1}%
   \section*{\homeworkProblemName}%
   \enterProblemHeader{\homeworkProblemName}}%
  {\exitProblemHeader{\homeworkProblemName}}%

\newcommand{\homeworkSectionName}{}%
\newlength{\homeworkSectionLabelLength}{}%
\newenvironment{homeworkSection}[1]%
  {% We put this space here to make sure we're not connected to the above.

   \renewcommand{\homeworkSectionName}{#1}%
   \settowidth{\homeworkSectionLabelLength}{\homeworkSectionName}%
   \addtolength{\homeworkSectionLabelLength}{0.25in}%
   \changetext{}{-\homeworkSectionLabelLength}{}{}{}%
   \subsection*{\homeworkSectionName}%
   \enterProblemHeader{\homeworkProblemName\ [\homeworkSectionName]}}%
  {\enterProblemHeader{\homeworkProblemName}%

   % We put the blank space above in order to make sure this margin
   % change doesn't happen too soon.
   \changetext{}{+\homeworkSectionLabelLength}{}{}{}}%

\newcommand{\Answer}{\ \\\textbf{Answer:} }
\newcommand{\Acknowledgement}[1]{\ \\{\bf Acknowledgement:} #1}

%%%%%%%%%%%%%%%%%%%%%%%%%%%%%%%%%%%%%%%%%%%%%%%%%%%%%%%%%%%%%


%%%%%%%%%%%%%%%%%%%%%%%%%%%%%%%%%%%%%%%%%%%%%%%%%%%%%%%%%%%%%
% Make title
\title{\textmd{\bf \Class: \Title}}
  \date{\textbf{\today}}
\author{\textbf{Qingru Hu \quad 2020012996}}
%%%%%%%%%%%%%%%%%%%%%%%%%%%%%%%%%%%%%%%%%%%%%%%%%%%%%%%%%%%%%

\begin{document}
\begin{spacing}{1.1}
\maketitle \thispagestyle{empty}

%%%%%%%%%%%%%%%%%%%%%%%%%%%%%%%%%%%%%%%%%%%%%%%%%%%%%%%%%%%%%
% Begin edit from here

\begin{homeworkProblem}

\begin{figure}[htbp]
  \centering
  \includegraphics[width=12cm]{cluster.jpg}
\end{figure}

\end{homeworkProblem}

\begin{homeworkProblem}
  I used the \texttt{sklearn.mixture.GaussianMixture} model to fit the data.
  This model uses EM algorithm to estimate the distribution. 
  The code for this problem is in \texttt{problem2.py}.
  The diagram for the prediction error with respect to iteration times $t$ 
  and total number of samples $N$ are shown in Fig.1 and Fig.2.

  \begin{figure}[htbp]
    \centering
    \includegraphics[width=10cm]{iterate_t.png}
    \caption{The prediction error with respect to $t$}
  \end{figure}

  \begin{figure}[htbp]
    \centering
    \includegraphics[width=10cm]{iterate_n.png}
    \caption{The prediction error with respect to $N$}
  \end{figure}

\end{homeworkProblem}

\begin{homeworkProblem}
  The code fo this problem is in \texttt{problem3.py}.
\section*{3.1}
I used the \texttt{sklearn.cluster.KMeans} model to do the clustering on training data points. 
The diagram of Je with respect to number of clusters is shown in Fig.3. 
The elbow point is around 15.
\begin{figure}[htbp]
  \centering
  \includegraphics[width=10cm]{Je.png}
  \caption{Je with respect to number of clusters}
\end{figure}

\section*{3.2}
The learned means of each cluster are stored in `kmeans' directory. They look like ten hand-written numbers.
I used 20 clusters in the prediction and the prediction accuracy on MNIST test dataset of \texttt{KMeans} is about 60\%.

\section*{3.3}
I used \texttt{sklearn.mixture.GaussianMixture} to use EM on MNIST. 
The learned means of each cluster are stored in `em' directory. 
I used 20 components in the prediction and the prediction accuracy on MNIST test dataset of \texttt{KMeans} is 
a little higher than 60\%.
\end{homeworkProblem}

% \newpage
\Acknowledgement{Thank Jingyuan Zhao (赵泾源) in this class for the discussion about Problem 2 and Problem 3.}

% End edit to here
%%%%%%%%%%%%%%%%%%%%%%%%%%%%%%%%%%%%%%%%%%%%%%%%%%%%%%%%%%%%%

\end{spacing}
\end{document}

%%%%%%%%%%%%%%%%%%%%%%%%%%%%%%%%%%%%%%%%%%%%%%%%%%%%%%%%%%%%%

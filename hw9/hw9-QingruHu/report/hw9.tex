% Modified based on Xiaoming Sun's template
\documentclass{article}
\usepackage{amsmath,amsfonts,amsthm,amssymb}
\usepackage{setspace}
\usepackage{fancyhdr}
\usepackage{lastpage}
\usepackage{extramarks}
\usepackage{chngpage}
\usepackage{soul,color}
\usepackage{graphicx,float,wrapfig}
\usepackage{enumitem}
\usepackage{array} 
\usepackage{hyperref}
\usepackage{float}
\usepackage{fontspec}
\setmonofont{Consolas}
\usepackage{listings}
\usepackage{xcolor}
\lstset{
  language = python, numbers=left, 
         numberstyle=\tiny,keywordstyle=\color{blue!70},
         commentstyle=\color{red!50!green!50!blue!50},frame=shadowbox,
         rulesepcolor=\color{red!20!green!20!blue!20},basicstyle=\ttfamily
}
\renewcommand{\d}{\mathrm{d}}
\newcommand{\Class}{Pattern Recognition and Machine Learning}

% Homework Specific Information. Change it to your own
\newcommand{\Title}{Homework 2}

% In case you need to adjust margins:
\topmargin=-0.45in      %
\evensidemargin=0in     %
\oddsidemargin=0in      %
\textwidth=6.5in        %
\textheight=9.0in       %
\headsep=0.25in         %

% Setup the header and footer
\pagestyle{fancy}                                                       %
\chead{\Title}  %
\rhead{\firstxmark}                                                     %
\lfoot{\lastxmark}                                                      %
\cfoot{}                                                                %
\rfoot{Page\ \thepage\ of\ \protect\pageref{LastPage}}                          %
\renewcommand\headrulewidth{0.4pt}                                      %
\renewcommand\footrulewidth{0.4pt}                                      %

%%%%%%%%%%%%%%%%%%%%%%%%%%%%%%%%%%%%%%%%%%%%%%%%%%%%%%%%%%%%%
% Some tools
\newcommand{\enterProblemHeader}[1]{\nobreak\extramarks{#1}{#1 continued on next page\ldots}\nobreak%
                                    \nobreak\extramarks{#1 (continued)}{#1 continued on next page\ldots}\nobreak}%
\newcommand{\exitProblemHeader}[1]{\nobreak\extramarks{#1 (continued)}{#1 continued on next page\ldots}\nobreak%
                                   \nobreak\extramarks{#1}{}\nobreak}%

\newcommand{\homeworkProblemName}{}%
\newcounter{homeworkProblemCounter}%
\newenvironment{homeworkProblem}[1][Problem \arabic{homeworkProblemCounter}]%
  {\stepcounter{homeworkProblemCounter}%
   \renewcommand{\homeworkProblemName}{#1}%
   \section*{\homeworkProblemName}%
   \enterProblemHeader{\homeworkProblemName}}%
  {\exitProblemHeader{\homeworkProblemName}}%

\newcommand{\homeworkSectionName}{}%
\newlength{\homeworkSectionLabelLength}{}%
\newenvironment{homeworkSection}[1]%
  {% We put this space here to make sure we're not connected to the above.

   \renewcommand{\homeworkSectionName}{#1}%
   \settowidth{\homeworkSectionLabelLength}{\homeworkSectionName}%
   \addtolength{\homeworkSectionLabelLength}{0.25in}%
   \changetext{}{-\homeworkSectionLabelLength}{}{}{}%
   \subsection*{\homeworkSectionName}%
   \enterProblemHeader{\homeworkProblemName\ [\homeworkSectionName]}}%
  {\enterProblemHeader{\homeworkProblemName}%

   % We put the blank space above in order to make sure this margin
   % change doesn't happen too soon.
   \changetext{}{+\homeworkSectionLabelLength}{}{}{}}%

\newcommand{\Answer}{\ \\\textbf{Answer:} }
\newcommand{\Acknowledgement}[1]{\ \\{\bf Acknowledgement:} #1}

%%%%%%%%%%%%%%%%%%%%%%%%%%%%%%%%%%%%%%%%%%%%%%%%%%%%%%%%%%%%%


%%%%%%%%%%%%%%%%%%%%%%%%%%%%%%%%%%%%%%%%%%%%%%%%%%%%%%%%%%%%%
% Make title
\title{\textmd{\bf \Class: \Title}}
  \date{\textbf{\today}}
\author{\textbf{Qingru Hu \quad 2020012996}}
%%%%%%%%%%%%%%%%%%%%%%%%%%%%%%%%%%%%%%%%%%%%%%%%%%%%%%%%%%%%%

\begin{document}
\begin{spacing}{1.1}
\maketitle \thispagestyle{empty}

%%%%%%%%%%%%%%%%%%%%%%%%%%%%%%%%%%%%%%%%%%%%%%%%%%%%%%%%%%%%%
% Begin edit from here

\begin{homeworkProblem}
The computing graph of the loss is as below.
\begin{figure}[htbp]
  \centering
  \includegraphics*[width=14cm]{graph.jpg}
\end{figure}

\end{homeworkProblem}


\begin{homeworkProblem}
  For the 2th layer:
\begin{align*}
  \frac{\partial J}{\partial y} &= -(z_2 - y)\\
  \frac{\partial J}{\partial z_2} &= z_2 - y\\
  \frac{\partial J}{\partial a_2} &= \frac{\partial z_2}{\partial a_2} \frac{\partial J}{\partial z_2} = W_2 \frac{\partial J}{\partial z_2}\\
  \frac{\partial J}{\partial W_2} &= \frac{\partial z_2}{\partial W_2} \frac{\partial J}{\partial z_2} = a_2 \frac{\partial J}{\partial z_2}\\
  \frac{\partial J}{\partial b_2} &= \frac{\partial z_2}{\partial b_2} \frac{\partial J}{\partial z_2} = \frac{\partial J}{\partial z_2}
\end{align*}

For the 1th layer:
\begin{align*}
  \frac{\partial J}{\partial z_1} &= \frac{\partial a_2}{\partial z_1} \frac{\partial J}{\partial a_2} = g'(z_1) \frac{\partial J}{\partial a_2} = H(z_1) \frac{\partial J}{\partial a_2}\\
  \frac{\partial J}{\partial a_1} &= \frac{\partial z_1}{\partial a_1} \frac{\partial J}{\partial z_1} = W_1 \frac{\partial J}{\partial z_1}\\
  \frac{\partial J}{\partial W_1} &= \frac{\partial z_1}{\partial W_1} \frac{\partial J}{\partial z_1} = a_1 \frac{\partial J}{\partial z_1}\\
  \frac{\partial J}{\partial b_1} &= \frac{\partial z_1}{\partial b_1} \frac{\partial J}{\partial z_1} = \frac{\partial J}{\partial z_1}
\end{align*}

For the 0th layer:
\begin{align*}
  \frac{\partial J}{\partial z_0} &= \frac{\partial a_1}{\partial z_0} \frac{\partial J}{\partial a_1} = g'(z_0) \frac{\partial J}{\partial a_1} = H(z_0) \frac{\partial J}{\partial a_1}\\
  \frac{\partial J}{\partial a_0} &= \frac{\partial z_0}{\partial a_0} \frac{\partial J}{\partial z_0} = W_0 \frac{\partial J}{\partial z_0} \\
  \frac{\partial J}{\partial W_0} &= \frac{\partial z_0}{\partial W_0} \frac{\partial J}{\partial z_0} = a_0 \frac{\partial J}{\partial z_0}\\
  \frac{\partial J}{\partial b_0} &= \frac{\partial z_0}{\partial b_0} \frac{\partial J}{\partial z_0} = \frac{\partial J}{\partial z_0} \\
  \frac{\partial J}{\partial x} &= \frac{\partial a_0}{\partial x} \frac{\partial J}{\partial a_0} = g'(x) \frac{\partial J}{\partial a_0} = H(x) \frac{\partial J}{\partial a_0} \\
\end{align*}
  
\end{homeworkProblem}

\begin{homeworkProblem}
  \begin{lstlisting}
# Define the function for students to implement back-propagation
def compute_gradient(x, y, W0, W1, W2, b0, b1, b2, a1, a2):
    a0 = relu(x)
    
    z2 = np.dot(a2, W2) + b2
    pz2 = (z2 - y) / x.shape[0]
    pa2 = pz2.dot(W2.T)
    pW2 = a2.T.dot(pz2)
    pb2 = np.sum(pz2)

    z1 = np.dot(a1, W1) + b1
    pz1 = relu_derivative(z1)*pa2
    pa1 = pz1.dot(W1.T)
    pW1 = a1.T.dot(pz1)
    pb1 = np.sum(pz1)
    
    z0 = np.dot(a0, W0) + b0
    pz0 = relu_derivative(z0)*(pa1)
    pW0 = a0.T.dot(pz0)
    pb0 = np.sum(pz0)
    
    return [pW0,pW1,pW2,pb0,pb1,pb2]
  \end{lstlisting}

The difference with the gradient computed from definition is shown as below.
\begin{figure}[htbp]
  \centering
  \includegraphics*[width=16cm]{diff.png}
\end{figure}

The test loss is 0.118030965884446.
\begin{lstlisting}
  0: loss is 7.166226621227566
1: loss is 5.436435240565046
2: loss is 4.362485487730143
3: loss is 3.5259562511192373
4: loss is 2.797636227865685
5: loss is 2.1391386688220955
6: loss is 1.554095074537603
7: loss is 1.0651384782714837
8: loss is 0.6967674140650253
9: loss is 0.4542093618541994
10: loss is 0.3156643620586208
...
90: loss is 0.09478981090562362
91: loss is 0.09442840347448693
92: loss is 0.09407267731409347
93: loss is 0.0937219001761906
94: loss is 0.09337641133499174
95: loss is 0.0930358066748061
96: loss is 0.09269980685026996
97: loss is 0.0923692779251972
98: loss is 0.0920438382813709
99: loss is 0.09172332782835861
Test loss is 0.118030965884446
\end{lstlisting}

\end{homeworkProblem}
% \newpage
% \Acknowledgement{Thank Siying Yang 2020012981 for
% the discussion about Problem 2.3 and Problem 3.}

% End edit to here
%%%%%%%%%%%%%%%%%%%%%%%%%%%%%%%%%%%%%%%%%%%%%%%%%%%%%%%%%%%%%

\end{spacing}
\end{document}

%%%%%%%%%%%%%%%%%%%%%%%%%%%%%%%%%%%%%%%%%%%%%%%%%%%%%%%%%%%%%

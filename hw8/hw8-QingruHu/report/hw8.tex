% Modified based on Xiaoming Sun's template
\documentclass{article}
\usepackage{amsmath,amsfonts,amsthm,amssymb}
\usepackage{setspace}
\usepackage{fancyhdr}
\usepackage{lastpage}
\usepackage{extramarks}
\usepackage{chngpage}
\usepackage{soul,color}
\usepackage{graphicx,float,wrapfig}
\usepackage{enumitem}
\usepackage{array} 
\usepackage{hyperref}
\usepackage{float}
\usepackage{fontspec}
\setmonofont{Consolas}
\usepackage{listings}
\usepackage{xcolor}
\lstset{
  language = python, numbers=left, 
         numberstyle=\tiny,keywordstyle=\color{blue!70},
         commentstyle=\color{red!50!green!50!blue!50},frame=shadowbox,
         rulesepcolor=\color{red!20!green!20!blue!20},basicstyle=\ttfamily
}
\renewcommand{\d}{\mathrm{d}}
\newcommand{\Class}{Pattern Recognition and Machine Learning}

% Homework Specific Information. Change it to your own
\newcommand{\Title}{Homework 2}

% In case you need to adjust margins:
\topmargin=-0.45in      %
\evensidemargin=0in     %
\oddsidemargin=0in      %
\textwidth=6.5in        %
\textheight=9.0in       %
\headsep=0.25in         %

% Setup the header and footer
\pagestyle{fancy}                                                       %
\chead{\Title}  %
\rhead{\firstxmark}                                                     %
\lfoot{\lastxmark}                                                      %
\cfoot{}                                                                %
\rfoot{Page\ \thepage\ of\ \protect\pageref{LastPage}}                          %
\renewcommand\headrulewidth{0.4pt}                                      %
\renewcommand\footrulewidth{0.4pt}                                      %

%%%%%%%%%%%%%%%%%%%%%%%%%%%%%%%%%%%%%%%%%%%%%%%%%%%%%%%%%%%%%
% Some tools
\newcommand{\enterProblemHeader}[1]{\nobreak\extramarks{#1}{#1 continued on next page\ldots}\nobreak%
                                    \nobreak\extramarks{#1 (continued)}{#1 continued on next page\ldots}\nobreak}%
\newcommand{\exitProblemHeader}[1]{\nobreak\extramarks{#1 (continued)}{#1 continued on next page\ldots}\nobreak%
                                   \nobreak\extramarks{#1}{}\nobreak}%

\newcommand{\homeworkProblemName}{}%
\newcounter{homeworkProblemCounter}%
\newenvironment{homeworkProblem}[1][Problem \arabic{homeworkProblemCounter}]%
  {\stepcounter{homeworkProblemCounter}%
   \renewcommand{\homeworkProblemName}{#1}%
   \section*{\homeworkProblemName}%
   \enterProblemHeader{\homeworkProblemName}}%
  {\exitProblemHeader{\homeworkProblemName}}%

\newcommand{\homeworkSectionName}{}%
\newlength{\homeworkSectionLabelLength}{}%
\newenvironment{homeworkSection}[1]%
  {% We put this space here to make sure we're not connected to the above.

   \renewcommand{\homeworkSectionName}{#1}%
   \settowidth{\homeworkSectionLabelLength}{\homeworkSectionName}%
   \addtolength{\homeworkSectionLabelLength}{0.25in}%
   \changetext{}{-\homeworkSectionLabelLength}{}{}{}%
   \subsection*{\homeworkSectionName}%
   \enterProblemHeader{\homeworkProblemName\ [\homeworkSectionName]}}%
  {\enterProblemHeader{\homeworkProblemName}%

   % We put the blank space above in order to make sure this margin
   % change doesn't happen too soon.
   \changetext{}{+\homeworkSectionLabelLength}{}{}{}}%

\newcommand{\Answer}{\ \\\textbf{Answer:} }
\newcommand{\Acknowledgement}[1]{\ \\{\bf Acknowledgement:} #1}

%%%%%%%%%%%%%%%%%%%%%%%%%%%%%%%%%%%%%%%%%%%%%%%%%%%%%%%%%%%%%


%%%%%%%%%%%%%%%%%%%%%%%%%%%%%%%%%%%%%%%%%%%%%%%%%%%%%%%%%%%%%
% Make title
\title{\textmd{\bf \Class: \Title}}
  \date{\textbf{\today}}
\author{\textbf{Qingru Hu \quad 2020012996}}
%%%%%%%%%%%%%%%%%%%%%%%%%%%%%%%%%%%%%%%%%%%%%%%%%%%%%%%%%%%%%

\begin{document}
\begin{spacing}{1.1}
\maketitle \thispagestyle{empty}

%%%%%%%%%%%%%%%%%%%%%%%%%%%%%%%%%%%%%%%%%%%%%%%%%%%%%%%%%%%%%
% Begin edit from here

\begin{homeworkProblem}
\subsection*{Task1}
\subsubsection*{1.}
Take the object function in Equation(2) into three parts:
\begin{align*}
  f_1 &= \frac{1}{l} \text{Tr}((Y-JKW)^{T}(Y-JKW)) \\
  f_2 &= \gamma_A \text{Tr}(W^T K W) \\
  f_3 &= \frac{\gamma_I}{(u+l)^2}\text{Tr}(W^TKLKW)
\end{align*}
Take the derivative of $f_1$:
\begin{align*}
  \d f_1 &= \frac{1}{l} \text{Tr} (\d (Y-JKW)^{T} (Y-JKW) + (Y-JKW)^{T} \d (Y-JKW)) \\
  &= \frac{2}{l} \text{Tr}((Y-JKW)^{T} \d (Y-JKW)) \\
  &= \frac{2}{l} \text{Tr}((Y-JKW)^{T} (-JK) \d W) \\
  \frac{\d f_1}{\d W} &= -\frac{2}{l} (JK)^{T} (Y-JKW)
\end{align*}
Take the derivative of $f_2$:
\begin{align*}
  \d f_2 &= \gamma_A \text{Tr} (\d W^T K W + W^T K \d W) \\
  &= 2 \gamma_A \text{Tr} (W^T K \d W) \\
  \frac{\d f_2}{\d W} &= 2 \gamma_A K W
\end{align*}
Take the derivative of $f_3$:
\begin{align*}
  \d f_3 &= \frac{\gamma_I}{(u+l)^2}\text{Tr}((\d W^T)KLKW + W^TKLK\d W) \\
  &= \frac{2\gamma_I}{(u+l)^2} \text{Tr}(W^TKLK\d W) \\
  \frac{\d f_3}{\d W} &= \frac{2\gamma_I}{(u+l)^2} KLK W
\end{align*}

\subsubsection*{2.}
\begin{lstlisting}
  self.W = np.linalg.inv(J.dot(K) + self.gamma_A*l*np.identity(l+u) 
           + (self.gamma_I*l)/(u+l)**2*L.dot(K)).dot(Y)
\end{lstlisting}

\subsection*{Task2}

\end{homeworkProblem}


% \newpage
% \Acknowledgement{Thank Siying Yang 2020012981 for
% the discussion about Problem 2.3 and Problem 3.}

% End edit to here
%%%%%%%%%%%%%%%%%%%%%%%%%%%%%%%%%%%%%%%%%%%%%%%%%%%%%%%%%%%%%

\end{spacing}
\end{document}

%%%%%%%%%%%%%%%%%%%%%%%%%%%%%%%%%%%%%%%%%%%%%%%%%%%%%%%%%%%%%
